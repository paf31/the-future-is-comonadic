\documentclass[nocopyrightspace]{sigplanconf}

\usepackage{amsmath}
\usepackage{amsthm}
\usepackage{hyperref}

\newcommand{\cL}{{\cal L}}

\begin{document}

\special{papersize=8.5in,11in}
\setlength{\pdfpageheight}{\paperheight}
\setlength{\pdfpagewidth}{\paperwidth}

\title{Declarative UIs are the Future --- And the Future is Comonadic!}

\authorinfo{Phil Freeman}
           {}
           {freeman.phil@gmail.com}

\maketitle

\begin{abstract}
There are many techniques for the declarative specification of user interfaces, but it is not
clear how to study their similarities and differences.
The category-theoretic notion of a \textit{comonad} captures some essential aspects of
these specification techniques.
The approach presented here generalizes several known techniques, but perhaps more interestingly,
it also generates several new comonads as we search for ways to represent existing user
interfaces.
\end{abstract}

\keywords
Haskell, functional programming, specification, laziness, user interfaces, comonads

\section{Motivation}

Here is a simple model of a user interface: a user interface consists of a type of states,
a value of that type which represents the initial state, and a function from states to views.
This data is captured precisely by the \texttt{Store} comonad:

\begin{verbatim}
data Store s a = Store
  { here :: s
  , view :: s -> a
  }
\end{verbatim}

We can move to a new state using a helper function:

\begin{verbatim}
move :: s -> Store s a -> Store s a
move s store = view (duplicate store) s
\end{verbatim}

\begin{figure}
\begin{verbatim}
class Functor w => Comonad w where
  extract :: w a -> a
  duplicate :: w a -> w (w a)

instance Comonad (Store s) where
  extract (Store here view) =
    view here
  duplicate (Store here view) =
    Store here (\next -> Store next view)
\end{verbatim}
\caption{The \texttt{Comonad} type class and \texttt{Store} instance}
\label{fig-comonad}
\end{figure}

The \texttt{duplicate} function is taken from the \texttt{Comonad} type class
(figure \ref{fig-comonad}).

A comonad represents a (lazy) unfolding of all possible future states of our user interface, as well as
the transitions allowed between those states. The \texttt{extract} function returns
a value for the current state, and \texttt{duplicate}
replaces each future state with its own structure of future states.

\section{Specifications from Comonads}

\begin{figure}
\begin{verbatim}
newtype Co w a = Co
  { runCo :: forall r. w (a -> r) -> r }

instance Comonad w => Monad (Co w)

select
  :: Comonad w => Co w (a -> b) -> w a -> w b
select co w = runCo co (extend dist w) where
  dist fs f = fmap (f $) fs
\end{verbatim}
\caption{The \texttt{Co w} monad for a \texttt{Comonad w} \citep{kmett}}
\label{fig-transition}
\end{figure}

\citet{kmett} defines a monad \texttt{Co w} which is
constructed from a comonad \texttt{w}. For our purposes, we
think of its actions as selecting some possible future state from a collection of
future states described by \texttt{w}. Figure \ref{fig-transition} defines the \texttt{Co w} monad, and
the \texttt{select} function which selects a future state.

We can now define a general framework for user interfaces.
A specification for a user interface will be described by a comonad \texttt{w} which describes
the type of reachable next states. The \texttt{Co w} monad for that comonad will
describe the transitions which are allowed.

We can use the \texttt{Co w} monad to understand the user interfaces that we
can describe using the comonad \texttt{w}.

An implementation would \texttt{extract} the current view at each state, and
user events would be associated with state transitions via the \texttt{Co w} monad.

\section{Examples}

In this section, we will interpret several known comonads in this new context, and
characterize the user interfaces that they describe. We will see that our approach
generalizes some known techniques.

\subsection{The Store Comonad}

We have already seen that the \texttt{Store s} comonad provides a useful model of
user interfaces with a state of type \texttt{s}.

The \texttt{Co (Store s)} monad is isomorphic to the usual \texttt{State s} monad,
providing full read/write access to the current state. In this
sense, the \texttt{Store s} comonad is a very unrestrictive model.

As an example, we can rewrite our \texttt{move} function from the previous section
in terms of \texttt{Co (Store s)}:

\begin{verbatim}
moveT :: s -> Co (Store s) ()
moveT s = Co (\w -> view w s ())
\end{verbatim}

\subsection{Moore Machines}

Moore machines form a comonad:

\begin{verbatim}
data Moore i a = Moore a (i -> Moore i a)
\end{verbatim}

Transitions are restricted --- in order to change state, the user
must provide an input of type \texttt{i}.

This approach is similar to the Elm architecture \citep{elm}.

\subsection{The Cofree Comonad}

Moore machines are a special case of a \textit{cofree comonad}:

\begin{verbatim}
data Cofree f a = Cofree a (f (Cofree f a))
\end{verbatim}

Here, transitions can be precisely described by the choice of some functor \texttt{f}.
It is possible to allow transitions limited read/write access to the current
state.

In fact, under certain conditions on \texttt{f}, the \texttt{Co (Cofree f)}
monad is isomorphic to a free monad which is determined by \texttt{f}.

This approach is reminiscent of the approach taken in the Halogen user interface
library \citep{halogen}.

\section{Composing Specifications}

In this section, we will work in the other direction, looking for comonads which
correspond to common patterns of composition in user interfaces.

\subsection{Sums}

Given two user interfaces, a common pattern is to display one or the other
at a time, and to allow the user to switch between the two.

To achieve this, we need to store all future states of both user interfaces, and an
additional bit of information to indicate which user interface is currently visible.
This data is packaged in the following data type, which we observe to be a \texttt{Comonad}:

\begin{verbatim}
data Sum f g a = Sum Bool (f a) (g a)
\end{verbatim}

\newtheorem{theorem}{Theorem}
\begin{theorem}
The \texttt{Sum} of two comonads is itself a comonad.
\end{theorem}

\begin{proof}
See \citet{freeman2}.
\end{proof}

\subsection{Day Convolution}

Another common pattern is to display two user interfaces side-by-side, with their
states evolving independently.

A comonad to specify such an interface would again need to store all future states of
both components, and we would also need a function to render any two current states.
This data is captured by the \textit{Day convolution} \citep{day} of the two functors,
expressed here as an \textit{existential data type}:

\begin{verbatim}
data Day f g a =
  forall x y. Day (x -> y -> a) (f x) (g y)
\end{verbatim}

The following result becomes necessary:

\begin{theorem}
The Day convolution of two comonads is itself a comonad.
\end{theorem}

\begin{proof}
See \citet{freeman1}.
\end{proof}

The \texttt{Co (Day f g)} monad for a Day convolution of comonads is difficult to
describe in general, but there exist natural transformations from \texttt{Co f}
and \texttt{Co g} to \texttt{Co (Day f g)} which embed transitions
for the individual components as transitions for the composition.

\subsection{Discussion}

Here, we started with existing, common user interface patterns and discovered
new comonads. What other idioms lead to useful \texttt{Comonad} instances?

The Day convolution gives the category of comonads the structure of a
\textit{symmetric monoidal category}. Can this be useful for studying user interfaces?

\section{Conclusion}

Comonads provide a way to generalize several approaches to the specification of
user interfaces, and compositions of comonads correspond to compositions
of those specifications.
This generalization suggests a principled approach to the design and
implementation of user interfaces, in which the specifications themselves become the objects
of study.

\acks

Many thanks to Edward Kmett and Jeff Polakow for providing useful feedback, and to
Arthur Xavier, for testing the ideas presented here.

\bibliographystyle{abbrvnat}

\begin{thebibliography}{10}
\softraggedright

\bibitem[Czaplicki(2013)]{elm}
E. Czaplicki, \emph{The Elm Architecture}, 2013
\newblock \url{guide.elm-lang.org/architecture}.

\bibitem[Day(1970)Day, Brian J.]{day}
Day, Brian J.
\newblock \emph{On closed categories of functors}, 1970
\newblock Reports of the Midwest Category Seminar IV, Springer Berlin Heidelberg, 1--38.

\bibitem[De Goes et~al.(2015)De Goes, Burgess]{halogen}
J. De Goes, G. Burgess, et al.,
\newblock \texttt{purescript-halogen}, 2015
\newblock \url{github.com/slamdata/purescript-halogen}.

\bibitem[Freeman(2016)Freeman]{freeman1}
Freeman, P., \emph{Comonads and Day Convolution}, 2016
\newblock \url{blog.functorial.com/posts/2016-08-08-Comonad-And-Day-Convolution.html}.

\bibitem[Freeman(2017)Freeman]{freeman2}
Freeman, P., \emph{Comonads for Optionality}, 2017
\newblock \url{blog.functorial.com/posts/2017-10-28-Comonads-For-Optionality.html}.

\bibitem[Kmett(2011)]{kmett}
E. Kmett, \emph{Monads from Comonads}, 2011
\newblock \url{comonad.com/reader/2011/monads-from-comonads}.

\end{thebibliography}
\end{document}
